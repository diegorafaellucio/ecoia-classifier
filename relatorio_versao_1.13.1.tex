\documentclass[12pt,a4paper]{article}
\usepackage[utf8]{inputenc}
\usepackage[brazilian]{babel}
\usepackage{hyperref}
\usepackage{graphicx}
\usepackage{listings}
\usepackage{xcolor}

\definecolor{codegreen}{rgb}{0,0.6,0}
\definecolor{codegray}{rgb}{0.5,0.5,0.5}
\definecolor{codepurple}{rgb}{0.58,0,0.82}
\definecolor{backcolour}{rgb}{0.95,0.95,0.92}

\lstdefinestyle{mystyle}{
    backgroundcolor=\color{backcolour},   
    commentstyle=\color{codegreen},
    keywordstyle=\color{magenta},
    numberstyle=\tiny\color{codegray},
    stringstyle=\color{codepurple},
    basicstyle=\ttfamily\footnotesize,
    breakatwhitespace=false,         
    breaklines=true,                 
    captionpos=b,                    
    keepspaces=true,                 
    numbers=left,                    
    numbersep=5pt,                  
    showspaces=false,                
    showstringspaces=false,
    showtabs=false,                  
    tabsize=2
}

\lstset{style=mystyle}

\title{Relatório de Atualização: ECOIA Classifier\\Versão 1.13.1}
\author{Equipe de Desenvolvimento}
\date{\today}

\begin{document}

\maketitle

\section{Introdução}

Este documento apresenta as alterações realizadas entre a versão 1.13.0 e a nova versão 1.13.1 do sistema ECOIA Classifier. O relatório detalha as modificações no código, novas funcionalidades, correções de bugs e melhorias de desempenho implementadas nesta atualização.

\section{Resumo das Alterações}

A versão 1.13.1 do ECOIA Classifier inclui as seguintes alterações principais:

\begin{itemize}
    \item Adição de um novo campo \texttt{cut\_intersection\_score} na tabela \texttt{bruise}
    \item Atualização da estrutura de arquivos do projeto
    \item Reorganização dos scripts de instalação e requisitos
    \item Melhorias no sistema de detecção de hematomas e sua relação com cortes
    \item Atualização de dependências para compatibilidade com Python 3.11
\end{itemize}

\section{Alterações no Banco de Dados}

Foi adicionado um novo campo na tabela \texttt{bruise}:

\begin{lstlisting}[language=SQL]
ALTER TABLE bruise
    ADD cut_intersection_score DECIMAL(10, 2) NULL;
\end{lstlisting}

Este campo armazena a pontuação de interseção entre o hematoma e o corte, permitindo uma análise mais precisa da relação entre hematomas e cortes específicos.

\section{Modificações no Código}

\subsection{Controlador de Hematomas}

O arquivo \texttt{src/controller/bruise\_controller.py} foi modificado para incluir o novo campo \texttt{cut\_intersection\_score} nos métodos de inserção de dados de hematomas:

\begin{lstlisting}[language=Python]
@staticmethod
def insert_into_bruise(image_id, bruise_id, cut_id, cut_intersection_score, bruise_coordinates, region_code_bruise, width=None, height=None, diameter=None, bruise_level_id=None):
    # Implementação atualizada para incluir o novo campo cut_intersection_score
\end{lstlisting}

\subsection{Utilitários de Hematomas}

O arquivo \texttt{src/utils/bruise\_utils.py} foi significativamente modificado:

\begin{itemize}
    \item O método \texttt{get\_id\_cuts\_affeted\_by\_bruises} foi renomeado para \texttt{get\_id\_intersection\_scores\_and\_cuts\_affected\_by\_bruises}
    \item O método agora retorna não apenas os IDs dos cortes afetados, mas também as pontuações de interseção
    \item A lógica de processamento foi atualizada para passar essas pontuações para o controlador de hematomas
\end{itemize}

\subsection{Handler do Integrador}

No arquivo \texttt{src/handler/integrator\_handler.py}, foi adicionada a recuperação do identificador do cliente a partir das configurações do sistema:

\begin{lstlisting}[language=Python]
client_identifier = ConfigurationStorageController.get_config_data_value(
    ConfigurationEnum.CLIENT_IDENTIFIER.name)
\end{lstlisting}

\section{Reorganização de Arquivos}

\subsection{Arquivos de Requisitos}

Os arquivos de requisitos foram reorganizados:

\begin{itemize}
    \item Os arquivos \texttt{data/install/requeriments\_base.txt} e \texttt{data/install/requeriments\_ultralytics.txt} foram movidos para o diretório raiz
    \item O arquivo \texttt{data/install/requeriments\_torch.txt} foi atualizado e movido para o diretório raiz
    \item As dependências foram atualizadas para compatibilidade com Python 3.11
\end{itemize}

\subsection{Scripts de Instalação}

O script de instalação \texttt{data/install/install.sh} foi substituído por um novo script \texttt{install.sh} no diretório raiz, com melhorias e atualizações para facilitar a configuração do ambiente.

\section{Arquivos Removidos}

Alguns arquivos foram removidos do projeto:

\begin{itemize}
    \item \texttt{data/scripts/add\_ip\_machine\_to\_env.py}
    \item \texttt{data/scripts/add\_watermask\_to\_old\_images.py}
    \item \texttt{test\_img.py}
\end{itemize}

\section{Conclusão}

A versão 1.13.1 do ECOIA Classifier traz melhorias significativas no sistema de detecção e análise de hematomas, especialmente na quantificação da relação entre hematomas e cortes específicos. A adição do campo \texttt{cut\_intersection\_score} permite uma análise mais precisa e detalhada, o que pode levar a insights mais valiosos sobre a qualidade da carne.

Além disso, a reorganização dos arquivos de requisitos e scripts de instalação simplifica o processo de configuração do ambiente.

Estas melhorias contribuem para um sistema mais robusto, preciso e fácil de usar, alinhado com os objetivos de qualidade e eficiência do ECOIA Classifier.

\end{document}
